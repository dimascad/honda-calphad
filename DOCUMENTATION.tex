\documentclass[11pt]{article}

% Packages
\usepackage[margin=1in]{geometry}
\usepackage{graphicx}
\usepackage{amsmath,amssymb}
\usepackage{booktabs}
\usepackage{array}
\usepackage{xcolor}
\usepackage{hyperref}
\usepackage{listings}
\usepackage{fancyhdr}
\usepackage{titlesec}
\usepackage{enumitem}
\usepackage{colortbl}

% Colors
\definecolor{OSUscarlet}{HTML}{BB0000}
\definecolor{OSUgray}{HTML}{666666}
\definecolor{codeblue}{HTML}{0077BB}
\definecolor{codeorange}{HTML}{EE7733}
\definecolor{codegray}{HTML}{F5F5F5}
\definecolor{tableheader}{HTML}{595959}

% Hyperref setup
\hypersetup{
    colorlinks=true,
    linkcolor=OSUscarlet,
    urlcolor=codeblue,
    citecolor=OSUscarlet
}

% Code listing style
\lstset{
    basicstyle=\ttfamily\small,
    backgroundcolor=\color{codegray},
    frame=single,
    framerule=0pt,
    breaklines=true,
    keywordstyle=\color{codeblue},
    commentstyle=\color{OSUgray},
    stringstyle=\color{codeorange},
    showstringspaces=false,
    tabsize=4
}

% Header/Footer
\pagestyle{fancy}
\fancyhf{}
\fancyhead[L]{\textcolor{OSUgray}{\small Honda CALPHAD Documentation}}
\fancyhead[R]{\textcolor{OSUgray}{\small MSE 4381}}
\fancyfoot[C]{\thepage}
\renewcommand{\headrulewidth}{0.4pt}

% Section formatting
\titleformat{\section}{\Large\bfseries\color{OSUscarlet}}{\thesection}{1em}{}
\titleformat{\subsection}{\large\bfseries}{\thesubsection}{1em}{}
\titleformat{\subsubsection}{\normalsize\bfseries}{\thesubsubsection}{1em}{}

\begin{document}

% Title
\begin{center}
    {\LARGE\bfseries How This Works: A Non-Coder's Guide}\\[0.5em]
    {\large Cu-Ceramic Thermodynamic Screening Tool}\\[1em]
    {\normalsize Anthony DiMascio}\\[0.5em]
    {\normalsize MSE 4381 Senior Design $\vert$ Honda CALPHAD Project $\vert$ Spring 2026}\\[0.5em]
    {\small Last updated: January 31, 2026}
\end{center}

\vspace{1em}
\hrule
\vspace{1.5em}

\tableofcontents
\newpage

%--------------------------------------------------
\section{What We Built}
%--------------------------------------------------

An \textbf{interactive thermodynamic screening tool} that helps answer: ``Which ceramic might best capture copper from molten steel?''

\subsection{The Files}

\begin{table}[h]
\centering
\renewcommand{\arraystretch}{1.3}
\begin{tabular}{>{\ttfamily}l p{8cm}}
\toprule
\rowcolor{tableheader}\textcolor{white}{\textbf{File}} & \textcolor{white}{\textbf{What it does}} \\
\midrule
cu\_ceramic\_thermodynamics.py & Main interactive notebook (Marimo) --- the one you share \\
\rowcolor{gray!15}
cu\_ceramic\_affinity.py & Standalone script that generates the Ellingham diagram PNG \\
pycalphad\_cu\_fe\_example.py & Demo showing how pyCALPHAD library works \\
\rowcolor{gray!15}
ellingham\_diagram.png & Static image of oxide stability comparison \\
cu\_fe\_gibbs\_energy.png & Gibbs energy curves for Cu-Fe system \\
\bottomrule
\end{tabular}
\end{table}

%--------------------------------------------------
\section{The Thermodynamics Explained}
%--------------------------------------------------

\subsection{Gibbs Free Energy ($\Delta G$)}

This is the ``will it react?'' number:
\begin{itemize}[noitemsep]
    \item $\Delta G < 0$ $\rightarrow$ Reaction happens spontaneously (favorable)
    \item $\Delta G > 0$ $\rightarrow$ Reaction won't happen on its own (unfavorable)
\end{itemize}

\subsection{How We Calculate It}

For oxide formation, we use a simple linear approximation:

\begin{equation}
\boxed{\Delta G_f^\circ = A + B \times T}
\end{equation}

Where:
\begin{itemize}[noitemsep]
    \item $A$ = enthalpy part (kJ/mol) --- roughly the ``heat'' of formation
    \item $B$ = entropy part (kJ/mol$\cdot$K) --- how ``disordered'' things get
    \item $T$ = temperature in Kelvin
\end{itemize}

\subsubsection{Example: Cu$_2$O}

The parameters for Cu$_2$O are: $A = -170$ kJ/mol, $B = 0.075$ kJ/mol$\cdot$K

At 1873 K (1600°C):
\begin{align}
\Delta G_f^\circ &= -170 + 0.075 \times 1873 \notag \\
&= -170 + 140.5 \notag \\
&= -29.5 \text{ kJ/mol}
\end{align}

\subsection{The Ellingham Diagram}

To compare different oxides fairly, we normalize everything to ``per mole of O$_2$'':

\begin{table}[h]
\centering
\renewcommand{\arraystretch}{1.3}
\begin{tabular}{llcc}
\toprule
\rowcolor{tableheader}
\textcolor{white}{\textbf{Oxide}} & \textcolor{white}{\textbf{Reaction}} & \textcolor{white}{\textbf{O$_2$ in rxn}} & \textcolor{white}{\textbf{Divide $\Delta G_f$ by}} \\
\midrule
Cu$_2$O & 2Cu + $\frac{1}{2}$O$_2$ $\rightarrow$ Cu$_2$O & 0.5 mol & 0.5 \\
\rowcolor{gray!15}
MgO & Mg + $\frac{1}{2}$O$_2$ $\rightarrow$ MgO & 0.5 mol & 0.5 \\
Al$_2$O$_3$ & 2Al + $\frac{3}{2}$O$_2$ $\rightarrow$ Al$_2$O$_3$ & 1.5 mol & 1.5 \\
\rowcolor{gray!15}
SiO$_2$ & Si + O$_2$ $\rightarrow$ SiO$_2$ & 1.0 mol & 1.0 \\
\bottomrule
\end{tabular}
\end{table}

\textbf{Key insight:} Lower on the diagram = more stable oxide.

\subsection{What the Diagram Tells Us}

Cu$_2$O is at the \textbf{TOP} (least stable). This means:
\begin{itemize}[noitemsep]
    \item Cu \textbf{cannot steal oxygen} from Al$_2$O$_3$, MgO, SiO$_2$, or TiO$_2$
    \item The reaction Cu + Al$_2$O$_3$ $\rightarrow$ Cu$_2$O + Al will \textbf{NOT} happen
    \item We need a \textit{different mechanism} to capture Cu
\end{itemize}

\subsection{The Actual Mechanisms}

Since Cu can't reduce the oxides, how does Al$_2$O$_3$ capture Cu? (It did in last year's experiments)

\begin{enumerate}[noitemsep]
    \item \textbf{Solid Solution} --- Cu atoms dissolve into the ceramic crystal at high T
    \item \textbf{Spinel Formation} --- CuAl$_2$O$_4$ forms (requires Cu to oxidize first)
    \item \textbf{Surface Adsorption} --- Cu sticks to particle surfaces
    \item \textbf{Capillary Action} --- Molten Cu wets and penetrates porous ceramics
\end{enumerate}

\subsection{Sulfide Exchange (The Alternative)}

Dr.\ Zhang showed this reaction works:
\begin{equation}
\text{2Cu} + \text{FeS} \rightarrow \text{Cu}_2\text{S} + \text{Fe}
\end{equation}

Why? Because Cu$_2$S is more stable than FeS ($\Delta G < 0$ for this reaction).

%--------------------------------------------------
\section{How the Code is Structured}
%--------------------------------------------------

\subsection{The Marimo Notebook Format}

Marimo notebooks are Python files with special structure:

\begin{lstlisting}[language=Python]
import marimo
app = marimo.App()

@app.cell          # This decorator marks a "cell"
def _():
    # Code goes here
    return variables_to_share

@app.cell
def _(variables_from_above):
    # This cell can use those variables
    return more_variables
\end{lstlisting}

\subsection{Key Parts of Our Notebook}

\subsubsection{1. Data Definition}

The oxide data is stored in a Python dictionary:

\begin{lstlisting}[language=Python]
oxide_data = {
    'Cu2O': (-170, 0.075, 0.5, '#0077BB', '-', '2Cu + 1/2O2 -> Cu2O'),
    #        A      B     O2_factor  color  linestyle  reaction
}
\end{lstlisting}

\subsubsection{2. Calculation Functions}

\begin{lstlisting}[language=Python]
def calc_dGf_per_O2(name, T_K):
    A, B, O2_factor, *_ = oxide_data[name]
    return (A + B * T_K) / O2_factor
\end{lstlisting}

\subsubsection{3. Interactive Controls}

\begin{lstlisting}[language=Python]
temp_slider = mo.ui.slider(start=1000, stop=1700, value=1600)
oxide_selector = mo.ui.multiselect(options=[...], value=[...])
\end{lstlisting}

%--------------------------------------------------
\section{How to Make Changes}
%--------------------------------------------------

\subsection{Change the Temperature Range}

Find this line and edit the numbers:
\begin{lstlisting}[language=Python]
temp_slider = mo.ui.slider(start=1000, stop=1700, ...)
\end{lstlisting}

\subsection{Add a New Oxide}

Add an entry to \texttt{oxide\_data}:
\begin{lstlisting}[language=Python]
'NewOxide': (A_value, B_value, O2_factor, '#HexColor', '-', 'Reaction'),
\end{lstlisting}

Where:
\begin{itemize}[noitemsep]
    \item Look up $A$ and $B$ values from thermodynamic tables (NIST-JANAF, Barin)
    \item O2\_factor = moles of O$_2$ in the formation reaction
    \item Color = hex code (colorblind-friendly: \texttt{\#0077BB}, \texttt{\#EE7733}, \texttt{\#009988})
\end{itemize}

\subsection{Change Label Positions}

Find the \texttt{\_min\_spacing} variable in the plot cell:
\begin{lstlisting}[language=Python]
_min_spacing = 45  # Increase for more space between labels
\end{lstlisting}

%--------------------------------------------------
\section{Limitations}
%--------------------------------------------------

\begin{center}
\fcolorbox{OSUscarlet}{yellow!20}{
\parbox{0.9\textwidth}{
\textbf{Warning:} This tool uses \textbf{simplified approximations}. It is a screening tool, not rigorous CALPHAD.
}}
\end{center}

\vspace{1em}

\begin{table}[h]
\centering
\renewcommand{\arraystretch}{1.3}
\begin{tabular}{p{5cm}p{7cm}}
\toprule
\rowcolor{tableheader}
\textcolor{white}{\textbf{Limitation}} & \textcolor{white}{\textbf{Why it matters}} \\
\midrule
No activity coefficients & Real solutions aren't ideal \\
\rowcolor{gray!15}
No phase diagrams & Can't find two-phase regions \\
No Cu solubility in oxides & This is the real mechanism! \\
\rowcolor{gray!15}
No kinetics & Doesn't tell you how fast \\
Linear $\Delta G$ approximation & Real curves have more terms \\
\bottomrule
\end{tabular}
\end{table}

\textbf{For real CALPHAD calculations, you need Thermo-Calc with proper databases.}

%--------------------------------------------------
\section{Next Steps: Thermo-Calc}
%--------------------------------------------------

\subsection{What Thermo-Calc Can Do That This Can't}

\begin{enumerate}[noitemsep]
    \item \textbf{Proper phase diagrams} --- Cu-Fe-O, Cu-Al-O ternary sections
    \item \textbf{Cu solubility} --- How much Cu dissolves in Al$_2$O$_3$ at 1600°C?
    \item \textbf{Activity coefficients} --- Real solution behavior
    \item \textbf{Equilibrium calculations} --- What phases are stable?
\end{enumerate}

\subsection{Databases You Need}

\begin{table}[h]
\centering
\renewcommand{\arraystretch}{1.3}
\begin{tabular}{lll}
\toprule
\rowcolor{tableheader}
\textcolor{white}{\textbf{Database}} & \textcolor{white}{\textbf{Contents}} & \textcolor{white}{\textbf{Use for}} \\
\midrule
TCFE & Steel thermodynamics & Cu in liquid Fe \\
\rowcolor{gray!15}
TCOX & Oxide systems & Cu-Al-O, Cu-Mg-O \\
SSUB & Pure substances & Reference data \\
\bottomrule
\end{tabular}
\end{table}

\subsection{Calculations to Run in Thermo-Calc}

\begin{enumerate}[noitemsep]
    \item \textbf{Cu-Fe binary phase diagram} --- Understand Cu miscibility in Fe
    \item \textbf{Cu activity in liquid Fe} --- At 1600°C, how ``active'' is 0.3\% Cu?
    \item \textbf{Cu-Al-O isothermal section at 1600°C} --- What phases form?
    \item \textbf{Property diagram: Cu solubility in Al$_2$O$_3$ vs T} --- The key data we need
\end{enumerate}

%--------------------------------------------------
\section{Workflow Summary}
%--------------------------------------------------

\begin{center}
\fbox{\parbox{0.85\textwidth}{
\centering
\textbf{This Notebook (Screening Tool)}\\
Quick oxide stability comparison\\
Shows Cu CAN'T reduce these oxides\\
Identifies need for different mechanism
}}

$\Big\downarrow$

\fbox{\parbox{0.85\textwidth}{
\centering
\textbf{Thermo-Calc (Rigorous CALPHAD)}\\
Calculate Cu solubility in each oxide\\
Model spinel formation conditions\\
Get real activity coefficients
}}

$\Big\downarrow$

\fbox{\parbox{0.85\textwidth}{
\centering
\textbf{Experiments (Fontana Lab)}\\
Test ceramics predicted to work best\\
Validate CALPHAD predictions\\
Measure actual Cu removal efficiency
}}
\end{center}

%--------------------------------------------------
\section{Reference: Thermodynamic Data}
%--------------------------------------------------

\subsection{Oxide Parameters Used}

\begin{table}[h]
\centering
\renewcommand{\arraystretch}{1.3}
\begin{tabular}{lcccc}
\toprule
\rowcolor{tableheader}
\textcolor{white}{\textbf{Oxide}} & \textcolor{white}{\textbf{$A$ (kJ/mol)}} & \textcolor{white}{\textbf{$B$ (kJ/mol$\cdot$K)}} & \textcolor{white}{\textbf{O$_2$ factor}} & \textcolor{white}{\textbf{Source}} \\
\midrule
Cu$_2$O & $-170$ & 0.075 & 0.5 & NIST-JANAF \\
\rowcolor{gray!15}
CuO & $-155$ & 0.085 & 0.5 & NIST-JANAF \\
FeO & $-264$ & 0.065 & 0.5 & NIST-JANAF \\
\rowcolor{gray!15}
Al$_2$O$_3$ & $-1676$ & 0.32 & 1.5 & NIST-JANAF \\
MgO & $-601$ & 0.11 & 0.5 & NIST-JANAF \\
\rowcolor{gray!15}
SiO$_2$ & $-910$ & 0.18 & 1.0 & NIST-JANAF \\
TiO$_2$ & $-944$ & 0.18 & 1.0 & NIST-JANAF \\
\bottomrule
\end{tabular}
\caption{Gibbs energy parameters for oxide formation ($\Delta G_f^\circ = A + BT$). Values are approximations.}
\end{table}

\subsection{Sulfide Parameters}

\begin{table}[h]
\centering
\renewcommand{\arraystretch}{1.3}
\begin{tabular}{lll}
\toprule
\rowcolor{tableheader}
\textcolor{white}{\textbf{Sulfide}} & \textcolor{white}{\textbf{$\Delta G_f$ approximation}} & \textcolor{white}{\textbf{Source}} \\
\midrule
FeS & $-150 + 0.027T$ & Literature estimate \\
\rowcolor{gray!15}
Cu$_2$S & $-180 + 0.032T$ & Literature estimate \\
\bottomrule
\end{tabular}
\caption{Gibbs energy parameters for sulfide formation.}
\end{table}

%--------------------------------------------------
\section{Links}
%--------------------------------------------------

\begin{itemize}[noitemsep]
    \item \textbf{GitHub Repository:} \url{https://github.com/dimascad/honda-calphad}
    \item \textbf{Interactive Notebook:} Open via Marimo from GitHub link
    \item \textbf{Marimo Documentation:} \url{https://docs.marimo.io}
\end{itemize}

\end{document}
