\documentclass[11pt]{article}
\usepackage[margin=1in]{geometry}
\usepackage{booktabs,array,xcolor,hyperref,listings,fancyhdr,enumitem,colortbl}
\usepackage{amsmath,amssymb}

% Colors
\definecolor{OSUscarlet}{HTML}{BB0000}
\definecolor{codeblue}{HTML}{0077BB}
\definecolor{codegray}{HTML}{F5F5F5}
\definecolor{tableheader}{HTML}{595959}

\hypersetup{colorlinks=true,linkcolor=OSUscarlet,urlcolor=codeblue}
\lstset{basicstyle=\ttfamily\small,backgroundcolor=\color{codegray},frame=single,framerule=0pt,breaklines=true,showstringspaces=false}

\pagestyle{fancy}
\fancyhf{}
\fancyhead[L]{\textcolor{gray}{\small Thermo-Calc Guide}}
\fancyhead[R]{\textcolor{gray}{\small Honda CALPHAD}}
\fancyfoot[C]{\thepage}

\setlength{\headheight}{14pt}

\begin{document}

\begin{center}
{\LARGE\bfseries Thermo-Calc Workflow Guide}\\[0.5em]
{\large Extracting CALPHAD Data for Python Plotting}\\[0.5em]
{\normalsize Anthony DiMascio}\\[0.3em]
{\small MSE 4381 $\vert$ Honda CALPHAD Project $\vert$ Spring 2026}
\end{center}
\vspace{0.5em}\hrule\vspace{1em}

%--------------------------------------------------
\section{File Organization}
%--------------------------------------------------

Create this folder structure \textbf{before} starting:

\begin{lstlisting}
honda-calphad/
  data/
    thermocalc/
      raw/           <- Direct exports from TC
      processed/     <- Cleaned CSVs for Python
    literature/      <- Reference values from papers
  simulations/
    pycalphad/       <- Your existing Marimo notebook
\end{lstlisting}

\subsection{File Naming Convention}

\begin{lstlisting}
[system]_[property]_[conditions].txt

Examples:
  cu2o_dGf_1273-1873K.txt      Gibbs energy of Cu2O formation
  al2o3_dGf_1273-1873K.txt     Gibbs energy of Al2O3 formation
  fe-cu_activity_1873K.txt     Cu activity in Fe at 1873K
  cu-al-o_isotherm_1873K.txt   Cu-Al-O phase diagram section
\end{lstlisting}

%--------------------------------------------------
\section{Calculation 1: Oxide Gibbs Energies (Ellingham Data)}
%--------------------------------------------------

\textbf{Goal:} Get accurate $\Delta G_f^\circ$ vs $T$ for each oxide to replace our approximations.

\subsection{Setup}

\begin{enumerate}[noitemsep]
\item Open Thermo-Calc $\rightarrow$ \texttt{File $\rightarrow$ New Project}
\item Select database: \textbf{SSUB} (SGTE Substance Database)
\item Define system: For Cu$_2$O, select elements \texttt{Cu, O}
\end{enumerate}

\subsection{For Each Oxide}

Repeat the calculation (Section 2.3--2.4) for each oxide below. Use the \textbf{exact mole fractions} shown to get stoichiometric composition.

\begin{table}[h]
\centering\small
\renewcommand{\arraystretch}{1.3}
\begin{tabular}{lllll}
\toprule
\rowcolor{tableheader}
\textcolor{white}{\textbf{Oxide}} & \textcolor{white}{\textbf{Elements}} & \textcolor{white}{\textbf{X (mole frac)}} & \textcolor{white}{\textbf{Phase}} & \textcolor{white}{\textbf{Output file}} \\
\midrule
Cu$_2$O & Cu, O & X(Cu)=0.6667, X(O)=0.3333 & CU2O\_S & \texttt{cu2o\_dGf.txt} \\
\rowcolor{gray!15}
CuO & Cu, O & X(Cu)=0.5, X(O)=0.5 & CUO\_S & \texttt{cuo\_dGf.txt} \\
Al$_2$O$_3$ & Al, O & X(Al)=0.4, X(O)=0.6 & CORUNDUM & \texttt{al2o3\_dGf.txt} \\
\rowcolor{gray!15}
MgO & Mg, O & X(Mg)=0.5, X(O)=0.5 & PERICLASE & \texttt{mgo\_dGf.txt} \\
SiO$_2$ & Si, O & X(Si)=0.3333, X(O)=0.6667 & QUARTZ$^*$ & \texttt{sio2\_dGf.txt} \\
\rowcolor{gray!15}
TiO$_2$ & Ti, O & X(Ti)=0.3333, X(O)=0.6667 & RUTILE & \texttt{tio2\_dGf.txt} \\
FeO & Fe, O & X(Fe)=0.5, X(O)=0.5 & HALITE$^{**}$ & \texttt{feo\_dGf.txt} \\
\bottomrule
\end{tabular}
\caption*{\footnotesize $^*$QUARTZ below 846K, CRISTOBALITE above; TC handles this automatically. $^{**}$FeO uses HALITE structure (NaCl-type) in SSUB.}
\end{table}

\subsubsection{Stoichiometry Reference}

The mole fractions come from the oxide formula:
\begin{align*}
\text{Cu}_2\text{O}: \quad & \frac{2}{2+1} = 0.6667 \text{ Cu}, \quad \frac{1}{2+1} = 0.3333 \text{ O} \\
\text{Al}_2\text{O}_3: \quad & \frac{2}{2+3} = 0.4 \text{ Al}, \quad \frac{3}{2+3} = 0.6 \text{ O} \\
\text{MO (CuO, MgO, FeO)}: \quad & \frac{1}{1+1} = 0.5 \text{ M}, \quad \frac{1}{1+1} = 0.5 \text{ O} \\
\text{MO}_2 \text{ (SiO}_2\text{, TiO}_2\text{)}: \quad & \frac{1}{1+2} = 0.3333 \text{ M}, \quad \frac{2}{1+2} = 0.6667 \text{ O}
\end{align*}

\subsection{Running the Calculation}

\textbf{In Thermo-Calc Graphical Mode:}

\begin{enumerate}[noitemsep]
\item \texttt{Calculate $\rightarrow$ Property Diagram}
\item \textbf{Axis Variable:}
    \begin{itemize}[noitemsep]
    \item Variable: \texttt{T} (Temperature)
    \item Min: \texttt{1273} K
    \item Max: \texttt{1873} K
    \item Step: \texttt{25} K (gives 25 data points)
    \end{itemize}
\item \textbf{Conditions:}
    \begin{itemize}[noitemsep]
    \item \texttt{P = 101325} (1 atm, in Pa)
    \item \texttt{N = 1} (1 mole total)
    \item \texttt{X(element)} = values from Table 1
    \end{itemize}
\item \textbf{Output:}
    \begin{itemize}[noitemsep]
    \item Add: \texttt{GM} (Gibbs energy of entire system, J/mol)
    \end{itemize}
\item Click \texttt{Calculate}
\end{enumerate}

\textbf{Console Mode (alternative):}
\begin{lstlisting}
go poly
s-sys Fe O
s-c t=1273 p=101325 n=1 x(fe)=0.5 x(o)=0.5
c-e
lis,,,,
\end{lstlisting}

\subsection{Exporting}

\begin{enumerate}[noitemsep]
\item \texttt{Results $\rightarrow$ Table Renderer}
\item Select columns: \texttt{T}, \texttt{GM}
\item \texttt{File $\rightarrow$ Export Table $\rightarrow$ Tab-separated}
\item Save to: \texttt{data/thermocalc/raw/cu2o\_dGf\_1273-1873K.txt}
\end{enumerate}

\textbf{Repeat for each oxide.} Total: 7 files.

%--------------------------------------------------
\section{Calculation 2: Cu Activity in Liquid Fe}
%--------------------------------------------------

\textbf{Goal:} How ``active'' is 0.3\% Cu in molten steel?

\subsection{Setup}
\begin{enumerate}[noitemsep]
\item Database: \textbf{TCFE} (Steels/Fe-alloys database)
\item \texttt{Define System}: select \texttt{Fe}, \texttt{Cu}
\item Reject all phases except \texttt{LIQUID} (we only want liquid steel)
\end{enumerate}

\subsection{Calculation A: Activity vs Composition at 1873K}

Shows how Cu activity changes with Cu content at fixed temperature.

\begin{enumerate}[noitemsep]
\item \texttt{Calculate $\rightarrow$ Property Diagram}
\item \textbf{Axis Variable:}
    \begin{itemize}[noitemsep]
    \item Variable: \texttt{X(Cu)}
    \item Min: \texttt{0.0001} (avoid zero)
    \item Max: \texttt{0.05} (5 mol\%)
    \item Step: \texttt{0.001}
    \end{itemize}
\item \textbf{Conditions:}
    \begin{itemize}[noitemsep]
    \item \texttt{T = 1873} K
    \item \texttt{P = 101325} Pa
    \item \texttt{N = 1}
    \end{itemize}
\item \textbf{Output:} \texttt{AC(Cu)} (activity of Cu, reference: pure liquid Cu)
\item Export: \texttt{fe-cu\_activity-vs-xcu\_1873K.txt}
\end{enumerate}

\subsection{Calculation B: Activity vs Temperature at 0.3\% Cu}

Shows how temperature affects Cu activity at typical contamination level.

\begin{enumerate}[noitemsep]
\item \texttt{Calculate $\rightarrow$ Property Diagram}
\item \textbf{Axis Variable:}
    \begin{itemize}[noitemsep]
    \item Variable: \texttt{T}
    \item Min: \texttt{1773} K (1500°C)
    \item Max: \texttt{1973} K (1700°C)
    \item Step: \texttt{10} K
    \end{itemize}
\item \textbf{Conditions:}
    \begin{itemize}[noitemsep]
    \item \texttt{X(Cu) = 0.003} (0.3 mol\% $\approx$ 0.3 wt\%)
    \item \texttt{P = 101325} Pa
    \item \texttt{N = 1}
    \end{itemize}
\item \textbf{Output:} \texttt{AC(Cu)}
\item Export: \texttt{fe-cu\_activity-vs-T\_xcu003.txt}
\end{enumerate}

%--------------------------------------------------
\section{Calculation 3: Cu-Al-O Phase Diagram}
%--------------------------------------------------

\textbf{Goal:} What phases exist when Cu contacts Al$_2$O$_3$? Are spinels (CuAl$_2$O$_4$) stable?

\subsection{Check Database Availability}

\begin{enumerate}[noitemsep]
\item In Thermo-Calc: \texttt{Database $\rightarrow$ Database Manager}
\item Look for: \textbf{TCOX} (Metal Oxide Solutions Database)
\item If unavailable: use \textbf{SSUB} (has stoichiometric oxides only, no solutions)
\end{enumerate}

\subsection{Setup (if TCOX available)}
\begin{enumerate}[noitemsep]
\item Database: \textbf{TCOX}
\item \texttt{Define System}: select \texttt{Cu}, \texttt{Al}, \texttt{O}
\item Keep all phases (to see what's stable)
\end{enumerate}

\subsection{Isothermal Section at 1873K}

\begin{enumerate}[noitemsep]
\item \texttt{Calculate $\rightarrow$ Phase Diagram $\rightarrow$ Ternary}
\item \textbf{Conditions:}
    \begin{itemize}[noitemsep]
    \item \texttt{T = 1873} K (fixed)
    \item \texttt{P = 101325} Pa (fixed)
    \end{itemize}
\item \textbf{Axes:} Cu-Al-O composition triangle (automatic)
\item Click \texttt{Calculate}
\item Look for:
    \begin{itemize}[noitemsep]
    \item CORUNDUM (Al$_2$O$_3$) --- does it have Cu solubility?
    \item SPINEL (CuAl$_2$O$_4$) --- is this phase stable?
    \item LIQUID regions --- where does Cu melt?
    \end{itemize}
\end{enumerate}

\subsection{Alternative: Pseudo-Binary Cu--Al$_2$O$_3$}

If ternary is complex, try a 1D cut:

\begin{enumerate}[noitemsep]
\item \texttt{Calculate $\rightarrow$ Property Diagram}
\item Fix O content at Al$_2$O$_3$ stoichiometry: X(O) = 0.6
\item Axis: vary X(Cu) from 0 to 0.3, with X(Al) = 0.4 $-$ X(Cu)$\times$(2/5)
\item This walks from pure Al$_2$O$_3$ toward Cu-rich compositions
\end{enumerate}

\textit{Note: If TCOX unavailable, this calculation is limited. Document what database was actually used.}

%--------------------------------------------------
\section{Calculation 4: Cu Solubility in Al$_2$O$_3$}
%--------------------------------------------------

\textbf{Goal:} Maximum Cu that dissolves in alumina vs temperature. This is the key data for predicting ceramic capture capacity.

\subsection{Check if Modeled}

First, check if CORUNDUM is a \textit{solution phase} or \textit{stoichiometric}:

\begin{enumerate}[noitemsep]
\item \texttt{Database $\rightarrow$ List Phase Status}
\item Find CORUNDUM
\item If it shows constituent species including Cu $\rightarrow$ solubility is modeled
\item If only Al:O $\rightarrow$ stoichiometric, no Cu solubility (limitation)
\end{enumerate}

\subsection{If Cu Solubility is Modeled}

\begin{enumerate}[noitemsep]
\item Database: \textbf{TCOX}
\item System: \texttt{Cu, Al, O}
\item \texttt{Calculate $\rightarrow$ Property Diagram}
\item \textbf{Axis:} \texttt{T} from 1273 to 1873 K, step 25 K
\item \textbf{Conditions:}
    \begin{itemize}[noitemsep]
    \item Set system in two-phase region: CORUNDUM + LIQUID or CORUNDUM + CU2O
    \item \texttt{P = 101325} Pa
    \end{itemize}
\item \textbf{Output:} \texttt{X(CORUNDUM,Cu)} --- mole fraction Cu in corundum phase
\item Export: \texttt{al2o3\_cu-solubility\_1273-1873K.txt}
\end{enumerate}

\subsection{If NOT Modeled (Stoichiometric)}

Document this limitation:
\begin{itemize}[noitemsep]
\item The database treats Al$_2$O$_3$ as a line compound with zero Cu solubility
\item Real Al$_2$O$_3$ may dissolve small amounts of Cu at high T (literature values needed)
\item Search literature for experimental Cu solubility data in Al$_2$O$_3$
\item Add to \texttt{data/literature/} folder
\end{itemize}

\subsection{Repeat for Other Ceramics}

If time permits, check Cu solubility in:
\begin{itemize}[noitemsep]
\item MgO (PERICLASE phase)
\item TiO$_2$ (RUTILE phase)
\item Spinels (SPINEL phase) --- may have higher Cu capacity
\end{itemize}

%--------------------------------------------------
\section{Data Processing}
%--------------------------------------------------

\subsection{Raw Export Format}

Thermo-Calc exports look like:
\begin{lstlisting}
T       GM
1273    -245678.3
1298    -246012.1
...
\end{lstlisting}

\subsection{Processing Script}

Save as \texttt{data/process\_tc\_data.py}:

\begin{lstlisting}[language=Python]
import numpy as np
import pandas as pd
from pathlib import Path

RAW = Path("thermocalc/raw")
OUT = Path("thermocalc/processed")
OUT.mkdir(exist_ok=True)

def process_oxide(filename, o2_factor, molar_mass_oxide):
    """Convert GM to dGf per mol O2 for Ellingham."""
    df = pd.read_csv(RAW/filename, sep='\t')
    df.columns = ['T_K', 'GM_J']

    # GM is in J/mol, convert to kJ/mol O2
    df['dGf_kJ_per_molO2'] = (df['GM_J'] / 1000) / o2_factor
    df['T_C'] = df['T_K'] - 273.15

    out_name = filename.replace('.txt', '_processed.csv')
    df.to_csv(OUT/out_name, index=False)
    print(f"Saved: {out_name}")

# Process each oxide
process_oxide('cu2o_dGf_1273-1873K.txt', 0.5, 143.09)
process_oxide('al2o3_dGf_1273-1873K.txt', 1.5, 101.96)
process_oxide('mgo_dGf_1273-1873K.txt', 0.5, 40.30)
process_oxide('sio2_dGf_1273-1873K.txt', 1.0, 60.08)
process_oxide('tio2_dGf_1273-1873K.txt', 1.0, 79.87)
process_oxide('feo_dGf_1273-1873K.txt', 0.5, 71.84)
\end{lstlisting}

%--------------------------------------------------
\section{Plotting in Python}
%--------------------------------------------------

Add to your Marimo notebook or create new script:

\begin{lstlisting}[language=Python]
import pandas as pd
import matplotlib.pyplot as plt
from pathlib import Path

DATA = Path("data/thermocalc/processed")

# Load all oxides
oxides = {
    'Cu2O':  ('cu2o_dGf_1273-1873K_processed.csv',  '#0077BB', '-'),
    'Al2O3': ('al2o3_dGf_1273-1873K_processed.csv', '#AA3377', ':'),
    'MgO':   ('mgo_dGf_1273-1873K_processed.csv',   '#009988', '-'),
    'SiO2':  ('sio2_dGf_1273-1873K_processed.csv',  '#CC3311', '--'),
    'TiO2':  ('tio2_dGf_1273-1873K_processed.csv',  '#E69F00', ':'),
    'FeO':   ('feo_dGf_1273-1873K_processed.csv',   '#EE7733', '--'),
}

fig, ax = plt.subplots(figsize=(10, 7))

for name, (file, color, ls) in oxides.items():
    df = pd.read_csv(DATA/file)
    ax.plot(df['T_C'], df['dGf_kJ_per_molO2'],
            color=color, ls=ls, lw=2.5, label=name)

ax.set_xlabel('Temperature (C)')
ax.set_ylabel('dG (kJ/mol O2)')
ax.set_title('Ellingham Diagram (Thermo-Calc Data)')
ax.legend()
ax.grid(True, alpha=0.3)
plt.tight_layout()
plt.savefig('ellingham_thermocalc.png', dpi=150)
\end{lstlisting}

%--------------------------------------------------
\section{Checklist}
%--------------------------------------------------

\begin{tabular}{@{}p{0.6\textwidth}l@{}}
\toprule
\rowcolor{tableheader}
\textcolor{white}{\textbf{Task}} & \textcolor{white}{\textbf{Done}} \\
\midrule
Create folder structure & $\square$ \\
\rowcolor{gray!15}
Export Cu$_2$O $\Delta G_f$ vs T & $\square$ \\
Export Al$_2$O$_3$ $\Delta G_f$ vs T & $\square$ \\
\rowcolor{gray!15}
Export MgO $\Delta G_f$ vs T & $\square$ \\
Export SiO$_2$ $\Delta G_f$ vs T & $\square$ \\
\rowcolor{gray!15}
Export TiO$_2$ $\Delta G_f$ vs T & $\square$ \\
Export FeO $\Delta G_f$ vs T & $\square$ \\
\rowcolor{gray!15}
Export Cu activity in Fe vs X(Cu) & $\square$ \\
Export Cu activity in Fe vs T & $\square$ \\
\rowcolor{gray!15}
Cu-Al-O isothermal section (if TCOX available) & $\square$ \\
Run processing script & $\square$ \\
\rowcolor{gray!15}
Generate new Ellingham plot & $\square$ \\
Update Marimo notebook with TC data & $\square$ \\
\bottomrule
\end{tabular}

\vspace{1em}
\hrule
\vspace{0.5em}
{\small\textit{After completing, commit data to GitHub:} \texttt{git add data/ \&\& git commit -m "Add TC data" \&\& git push}}

\end{document}
