\documentclass[11pt]{article}
\usepackage[margin=1in]{geometry}
\usepackage{booktabs,array,xcolor,hyperref,listings,fancyhdr,enumitem,colortbl}
\usepackage{amsmath,amssymb}

% Colors
\definecolor{OSUscarlet}{HTML}{BB0000}
\definecolor{codeblue}{HTML}{0077BB}
\definecolor{codegray}{HTML}{F5F5F5}
\definecolor{tableheader}{HTML}{595959}

\hypersetup{colorlinks=true,linkcolor=OSUscarlet,urlcolor=codeblue}
\lstset{basicstyle=\ttfamily\small,backgroundcolor=\color{codegray},frame=single,framerule=0pt,breaklines=true,showstringspaces=false}

\pagestyle{fancy}
\fancyhf{}
\fancyhead[L]{\textcolor{gray}{\small Thermo-Calc Guide}}
\fancyhead[R]{\textcolor{gray}{\small Honda CALPHAD}}
\fancyfoot[C]{\thepage}

\setlength{\headheight}{14pt}

\begin{document}

\begin{center}
{\LARGE\bfseries Thermo-Calc Workflow Guide}\\[0.5em]
{\large Extracting CALPHAD Data for Python Plotting}\\[0.5em]
{\normalsize Anthony DiMascio}\\[0.3em]
{\small MSE 4381 $\vert$ Honda CALPHAD Project $\vert$ Spring 2026}
\end{center}
\vspace{0.5em}\hrule\vspace{1em}

%--------------------------------------------------
\section{File Organization}
%--------------------------------------------------

Create this folder structure \textbf{before} starting:

\begin{lstlisting}
honda-calphad/
  data/
    thermocalc/
      raw/           <- Direct exports from TC
      processed/     <- Cleaned CSVs for Python
    literature/      <- Reference values from papers
  simulations/
    pycalphad/       <- Your existing Marimo notebook
\end{lstlisting}

\subsection{File Naming Convention}

\begin{lstlisting}
[system]_[property]_[conditions].txt

Examples:
  cu2o_dGf_1273-1873K.txt      Gibbs energy of Cu2O formation
  al2o3_dGf_1273-1873K.txt     Gibbs energy of Al2O3 formation
  fe-cu_activity_1873K.txt     Cu activity in Fe at 1873K
  cu-al-o_isotherm_1873K.txt   Cu-Al-O phase diagram section
\end{lstlisting}

%--------------------------------------------------
\section{Calculation 1: Oxide Gibbs Energies (Ellingham Data)}
%--------------------------------------------------

\textbf{Goal:} Get accurate $\Delta G_f^\circ$ vs $T$ for each oxide to replace our approximations.

\subsection{Setup}

\begin{enumerate}[noitemsep]
\item Open Thermo-Calc $\rightarrow$ \texttt{File $\rightarrow$ New Project}
\item Select database: \textbf{SSUB} (SGTE Substance Database)
\item Define system: For Cu$_2$O, select elements \texttt{Cu, O}
\end{enumerate}

\subsection{For Each Oxide}

Repeat the calculation (Section 2.3--2.4) for each oxide below. Change the elements and look for the corresponding phase name in the database.

\begin{table}[h]
\centering\small
\renewcommand{\arraystretch}{1.2}
\begin{tabular}{llll}
\toprule
\rowcolor{tableheader}
\textcolor{white}{\textbf{Oxide}} & \textcolor{white}{\textbf{Elements}} & \textcolor{white}{\textbf{Phase to check}} & \textcolor{white}{\textbf{Output file}} \\
\midrule
Cu$_2$O & Cu, O & CU2O\_S & \texttt{cu2o\_dGf\_1273-1873K.txt} \\
\rowcolor{gray!15}
CuO & Cu, O & CUO\_S & \texttt{cuo\_dGf\_1273-1873K.txt} \\
Al$_2$O$_3$ & Al, O & CORUNDUM & \texttt{al2o3\_dGf\_1273-1873K.txt} \\
\rowcolor{gray!15}
MgO & Mg, O & PERICLASE & \texttt{mgo\_dGf\_1273-1873K.txt} \\
SiO$_2$ & Si, O & QUARTZ / CRISTOBALITE & \texttt{sio2\_dGf\_1273-1873K.txt} \\
\rowcolor{gray!15}
TiO$_2$ & Ti, O & RUTILE & \texttt{tio2\_dGf\_1273-1873K.txt} \\
FeO & Fe, O & WUSTITE & \texttt{feo\_dGf\_1273-1873K.txt} \\
\bottomrule
\end{tabular}
\end{table}

\subsection{Running the Calculation}

\begin{enumerate}[noitemsep]
\item \texttt{Equilibrium Calculator $\rightarrow$ Property Diagram}
\item Axis variable: \textbf{T} from 1273 to 1873 K, step 25 K
\item Set composition to stoichiometric oxide (e.g., 66.67\% Cu, 33.33\% O for Cu$_2$O)
\item Add to plot: \texttt{GM(*)} (Gibbs energy of system)
\item \texttt{Perform $\rightarrow$ Calculate}
\end{enumerate}

\subsection{Exporting}

\begin{enumerate}[noitemsep]
\item \texttt{Results $\rightarrow$ Table Renderer}
\item Select columns: \texttt{T}, \texttt{GM}
\item \texttt{File $\rightarrow$ Export Table $\rightarrow$ Tab-separated}
\item Save to: \texttt{data/thermocalc/raw/cu2o\_dGf\_1273-1873K.txt}
\end{enumerate}

\textbf{Repeat for each oxide.} Total: 7 files.

%--------------------------------------------------
\section{Calculation 2: Cu Activity in Liquid Fe}
%--------------------------------------------------

\textbf{Goal:} How ``active'' is 0.3\% Cu in molten steel?

\subsection{Setup}
\begin{enumerate}[noitemsep]
\item Database: \textbf{TCFE} (Steels/Fe-alloys)
\item System: \texttt{Fe, Cu}
\item Phase: \texttt{LIQUID}
\end{enumerate}

\subsection{Calculation A: Activity vs Composition at 1873K}
\begin{enumerate}[noitemsep]
\item Property Diagram, axis: \texttt{X(Cu)} from 0 to 0.05 (0--5\%)
\item Fixed: $T = 1873$ K
\item Plot: \texttt{AC(Cu)} (activity of Cu)
\item Export: \texttt{fe-cu\_activity-vs-xcu\_1873K.txt}
\end{enumerate}

\subsection{Calculation B: Activity vs Temperature at 0.3\% Cu}
\begin{enumerate}[noitemsep]
\item Property Diagram, axis: \texttt{T} from 1773 to 1973 K
\item Fixed: \texttt{X(Cu)} = 0.003
\item Plot: \texttt{AC(Cu)}
\item Export: \texttt{fe-cu\_activity-vs-T\_xcu003.txt}
\end{enumerate}

%--------------------------------------------------
\section{Calculation 3: Cu-Al-O Phase Diagram}
%--------------------------------------------------

\textbf{Goal:} What phases exist when Cu contacts Al$_2$O$_3$?

\subsection{Setup}
\begin{enumerate}[noitemsep]
\item Database: \textbf{TCOX} (Oxides) --- \textit{check if available}
\item System: \texttt{Cu, Al, O}
\end{enumerate}

\subsection{Isothermal Section at 1873K}
\begin{enumerate}[noitemsep]
\item \texttt{Phase Diagram $\rightarrow$ Ternary}
\item Fix $T = 1873$ K, $P = 1$ atm
\item Generate diagram
\item Export tie-lines/phase boundaries as table
\item Save: \texttt{cu-al-o\_isotherm\_1873K.txt}
\end{enumerate}

\textit{Note: If TCOX unavailable, check SSUB for limited oxide data.}

%--------------------------------------------------
\section{Calculation 4: Cu Solubility in Al$_2$O$_3$}
%--------------------------------------------------

\textbf{Goal:} Maximum Cu that dissolves in alumina vs temperature.

This requires checking if the database models Cu solubility in the CORUNDUM phase.

\begin{enumerate}[noitemsep]
\item System: Cu-Al-O, database TCOX
\item Fix Al$_2$O$_3$ composition (Al:O = 2:3)
\item Step temperature 1273--1873 K
\item Check: \texttt{X(CORUNDUM,Cu)} if available
\item Export: \texttt{al2o3\_cu-solubility\_1273-1873K.txt}
\end{enumerate}

\textit{If not modeled:} Note this as a limitation; the database may treat Al$_2$O$_3$ as a stoichiometric compound with no Cu solubility.

%--------------------------------------------------
\section{Data Processing}
%--------------------------------------------------

\subsection{Raw Export Format}

Thermo-Calc exports look like:
\begin{lstlisting}
T       GM
1273    -245678.3
1298    -246012.1
...
\end{lstlisting}

\subsection{Processing Script}

Save as \texttt{data/process\_tc\_data.py}:

\begin{lstlisting}[language=Python]
import numpy as np
import pandas as pd
from pathlib import Path

RAW = Path("thermocalc/raw")
OUT = Path("thermocalc/processed")
OUT.mkdir(exist_ok=True)

def process_oxide(filename, o2_factor, molar_mass_oxide):
    """Convert GM to dGf per mol O2 for Ellingham."""
    df = pd.read_csv(RAW/filename, sep='\t')
    df.columns = ['T_K', 'GM_J']

    # GM is in J/mol, convert to kJ/mol O2
    df['dGf_kJ_per_molO2'] = (df['GM_J'] / 1000) / o2_factor
    df['T_C'] = df['T_K'] - 273.15

    out_name = filename.replace('.txt', '_processed.csv')
    df.to_csv(OUT/out_name, index=False)
    print(f"Saved: {out_name}")

# Process each oxide
process_oxide('cu2o_dGf_1273-1873K.txt', 0.5, 143.09)
process_oxide('al2o3_dGf_1273-1873K.txt', 1.5, 101.96)
process_oxide('mgo_dGf_1273-1873K.txt', 0.5, 40.30)
process_oxide('sio2_dGf_1273-1873K.txt', 1.0, 60.08)
process_oxide('tio2_dGf_1273-1873K.txt', 1.0, 79.87)
process_oxide('feo_dGf_1273-1873K.txt', 0.5, 71.84)
\end{lstlisting}

%--------------------------------------------------
\section{Plotting in Python}
%--------------------------------------------------

Add to your Marimo notebook or create new script:

\begin{lstlisting}[language=Python]
import pandas as pd
import matplotlib.pyplot as plt
from pathlib import Path

DATA = Path("data/thermocalc/processed")

# Load all oxides
oxides = {
    'Cu2O':  ('cu2o_dGf_1273-1873K_processed.csv',  '#0077BB', '-'),
    'Al2O3': ('al2o3_dGf_1273-1873K_processed.csv', '#AA3377', ':'),
    'MgO':   ('mgo_dGf_1273-1873K_processed.csv',   '#009988', '-'),
    'SiO2':  ('sio2_dGf_1273-1873K_processed.csv',  '#CC3311', '--'),
    'TiO2':  ('tio2_dGf_1273-1873K_processed.csv',  '#E69F00', ':'),
    'FeO':   ('feo_dGf_1273-1873K_processed.csv',   '#EE7733', '--'),
}

fig, ax = plt.subplots(figsize=(10, 7))

for name, (file, color, ls) in oxides.items():
    df = pd.read_csv(DATA/file)
    ax.plot(df['T_C'], df['dGf_kJ_per_molO2'],
            color=color, ls=ls, lw=2.5, label=name)

ax.set_xlabel('Temperature (C)')
ax.set_ylabel('dG (kJ/mol O2)')
ax.set_title('Ellingham Diagram (Thermo-Calc Data)')
ax.legend()
ax.grid(True, alpha=0.3)
plt.tight_layout()
plt.savefig('ellingham_thermocalc.png', dpi=150)
\end{lstlisting}

%--------------------------------------------------
\section{Checklist}
%--------------------------------------------------

\begin{tabular}{@{}p{0.6\textwidth}l@{}}
\toprule
\rowcolor{tableheader}
\textcolor{white}{\textbf{Task}} & \textcolor{white}{\textbf{Done}} \\
\midrule
Create folder structure & $\square$ \\
\rowcolor{gray!15}
Export Cu$_2$O $\Delta G_f$ vs T & $\square$ \\
Export Al$_2$O$_3$ $\Delta G_f$ vs T & $\square$ \\
\rowcolor{gray!15}
Export MgO $\Delta G_f$ vs T & $\square$ \\
Export SiO$_2$ $\Delta G_f$ vs T & $\square$ \\
\rowcolor{gray!15}
Export TiO$_2$ $\Delta G_f$ vs T & $\square$ \\
Export FeO $\Delta G_f$ vs T & $\square$ \\
\rowcolor{gray!15}
Export Cu activity in Fe vs X(Cu) & $\square$ \\
Export Cu activity in Fe vs T & $\square$ \\
\rowcolor{gray!15}
Cu-Al-O isothermal section (if TCOX available) & $\square$ \\
Run processing script & $\square$ \\
\rowcolor{gray!15}
Generate new Ellingham plot & $\square$ \\
Update Marimo notebook with TC data & $\square$ \\
\bottomrule
\end{tabular}

\vspace{1em}
\hrule
\vspace{0.5em}
{\small\textit{After completing, commit data to GitHub:} \texttt{git add data/ \&\& git commit -m "Add TC data" \&\& git push}}

\end{document}
