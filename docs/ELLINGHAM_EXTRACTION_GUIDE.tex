\documentclass[11pt]{article}
\usepackage[margin=1in]{geometry}
\usepackage{graphicx}
\usepackage{amsmath}
\usepackage{amssymb}
\usepackage{booktabs}
\usepackage{array}
\usepackage[table]{xcolor}
\usepackage{hyperref}
\usepackage{fancyhdr}
\usepackage{listings}
\usepackage{tcolorbox}

% Colors
\definecolor{scarlet}{RGB}{187,0,0}
\definecolor{gray}{RGB}{102,102,102}
\definecolor{codebg}{RGB}{245,245,245}
\definecolor{codeframe}{RGB}{200,200,200}

% Header/Footer
\pagestyle{fancy}
\fancyhf{}
\fancyhead[L]{\textcolor{scarlet}{\textbf{Honda CALPHAD Project}}}
\fancyhead[R]{\textcolor{gray}{MSE 4381 | Spring 2026}}
\fancyfoot[C]{\thepage}
\renewcommand{\headrulewidth}{0.4pt}

% Code listing style
\lstset{
    backgroundcolor=\color{codebg},
    frame=single,
    rulecolor=\color{codeframe},
    basicstyle=\ttfamily\small,
    breaklines=true,
    columns=fullflexible
}

% Custom box for key findings
\newtcolorbox{keybox}[1][]{
    colback=scarlet!5,
    colframe=scarlet!75,
    fonttitle=\bfseries,
    title=#1
}

\begin{document}

% Title
\begin{center}
    {\LARGE\textbf{Ellingham Diagram Data Extraction}}\\[0.3cm]
    {\Large\textcolor{scarlet}{Technical Guide}}\\[0.5cm]
    {\large Honda CALPHAD Project | MSE 4381 Senior Design}\\[0.3cm]
    {\normalsize February 4, 2026}
\end{center}

\vspace{0.5cm}

\tableofcontents
\newpage

% ============================================================================
\section{Introduction}
% ============================================================================

\subsection{Objective}
Extract Gibbs energies of formation ($\Delta G_f^\circ$) for oxide materials from Thermo-Calc's TCOX14 database to create an Ellingham diagram. This diagram compares oxide stability and answers a key question for the Cu removal project:

\begin{keybox}[Research Question]
Can copper thermodynamically reduce ceramic oxides (Al$_2$O$_3$, MgO, SiO$_2$, TiO$_2$)?
\end{keybox}

\subsection{What is an Ellingham Diagram?}
An Ellingham diagram plots $\Delta G_f^\circ$ per mole O$_2$ versus temperature for metal oxide formation reactions.

\textbf{Key principles:}
\begin{itemize}
    \item Lower (more negative) $\Delta G$ = more stable oxide
    \item A metal can reduce another metal's oxide only if its oxide line is \textit{below} the other
    \item The vertical gap between lines indicates the thermodynamic driving force
\end{itemize}

% ============================================================================
\section{Method}
% ============================================================================

\subsection{Software and Database}
\begin{itemize}
    \item \textbf{Thermo-Calc 2025b} with \textbf{TC-Python} API
    \item \textbf{TCOX14} database (comprehensive oxide thermodynamics)
    \item Run on \textbf{OSU ETS Virtual Machine} (Windows, Thermo-Calc license)
\end{itemize}

\subsection{Oxides Calculated}

\begin{table}[h]
\centering
\caption{Oxide phases and formation reactions normalized per mole O$_2$}
\begin{tabular}{llll}
\toprule
\textbf{Oxide} & \textbf{Mineral Name} & \textbf{TC Phase} & \textbf{Reaction (per mol O$_2$)} \\
\midrule
Cu$_2$O & Cuprite & CUPRITE\#1 & 4Cu + O$_2$ $\rightarrow$ 2Cu$_2$O \\
CuO & Tenorite & CUO\#1 & 2Cu + O$_2$ $\rightarrow$ 2CuO \\
Al$_2$O$_3$ & Corundum & CORUNDUM\#1 & $\frac{4}{3}$Al + O$_2$ $\rightarrow$ $\frac{2}{3}$Al$_2$O$_3$ \\
MgO & Periclase & HALITE\#1 & 2Mg + O$_2$ $\rightarrow$ 2MgO \\
SiO$_2$ & Quartz & QUARTZ\#1 & Si + O$_2$ $\rightarrow$ SiO$_2$ \\
TiO$_2$ & Rutile & RUTILE\#1 & Ti + O$_2$ $\rightarrow$ TiO$_2$ \\
FeO & W\"ustite & HALITE\#1 & 2Fe + O$_2$ $\rightarrow$ 2FeO \\
\bottomrule
\end{tabular}
\end{table}

\subsection{Temperature Range}
\begin{itemize}
    \item Range: 500 K to 2000 K
    \item Step size: 50 K
    \item Total: 31 data points per oxide
\end{itemize}

% ============================================================================
\section{Calculation Details}
% ============================================================================

\subsection{Formation Energy Formula}
For the general reaction: Metal + O$_2$ $\rightarrow$ Oxide

\begin{equation}
\Delta G_f^\circ = G_{oxide}^\circ - n \cdot G_{metal}^\circ - G_{O_2}^\circ
\end{equation}

Where:
\begin{itemize}
    \item $G_{oxide}^\circ$ = Gibbs energy of the oxide phase (per formula unit)
    \item $G_{metal}^\circ$ = Gibbs energy of pure metal reference state
    \item $G_{O_2}^\circ$ = Gibbs energy of O$_2$ gas at 1 atm
    \item $n$ = stoichiometric coefficient for metal
\end{itemize}

\subsection{Normalization per Mole O$_2$}
All values are normalized per mole O$_2$ consumed for direct comparison on the Ellingham diagram. For example:
\begin{itemize}
    \item \textbf{Cu$_2$O:} $\Delta G_{per\ O_2} = 2 \times G(Cu_2O) - 4 \times G(Cu) - G(O_2)$
    \item \textbf{Al$_2$O$_3$:} $\Delta G_{per\ O_2} = \frac{2}{3} \times G(Al_2O_3) - \frac{4}{3} \times G(Al) - G(O_2)$
\end{itemize}

\subsection{TC-Python Implementation}

The extraction script performs four steps at each temperature:

\begin{enumerate}
    \item \textbf{O$_2$ Reference:} Calculate $G(O_2)$ using pure O system
    \begin{itemize}
        \item TC returns $GM$ per mole O atoms $\rightarrow$ multiply by 2 for per mole O$_2$
    \end{itemize}

    \item \textbf{Metal Reference:} Calculate $G(metal)$ at near-zero oxygen ($X_O = 0.0001$)

    \item \textbf{Oxide Phase:} Set stoichiometric composition, calculate equilibrium
    \begin{itemize}
        \item Extract individual phase energy: \texttt{GM(CUPRITE\#1)}
        \item Convert per-atom to per-formula-unit: multiply by atoms per formula
    \end{itemize}

    \item \textbf{Formation Energy:} Apply the $\Delta G_f$ formula
\end{enumerate}

\subsection{Technical Challenges Solved}

\begin{table}[h]
\centering
\caption{Bugs encountered and solutions implemented}
\begin{tabular}{p{4cm}p{5cm}p{5cm}}
\toprule
\textbf{Problem} & \textbf{Cause} & \textbf{Solution} \\
\midrule
Wrong sign for Cu$_2$O (+97 instead of -191 kJ/mol) & Two-phase equilibrium (CUPRITE + FCC\_A1); system $GM$ is mixture energy & Extract individual phase energy using \texttt{GM(CUPRITE\#1)} \\
\addlinespace
Values too small by factor of 2-5 & TC returns $GM$ per mole atoms, not per formula unit & Multiply by atoms\_per\_formula (e.g., 3 for Cu$_2$O) \\
\addlinespace
O$_2$ reference off by factor of 2 & $GM$ for pure O is per mole O atoms & Multiply by 2 for per mole O$_2$ \\
\bottomrule
\end{tabular}
\end{table}

% ============================================================================
\section{How the Pipeline Works}
% ============================================================================

\subsection{What the Script Actually Does}

The extraction script is not doing anything clever. It is a mechanical loop that calls the same Thermo-Calc API function over and over with different inputs. At each temperature, for each oxide, the script asks Thermo-Calc three questions:

\begin{enumerate}
    \item What is the Gibbs energy of the \textbf{pure metal}? (e.g., pure Cu)
    \item What is the Gibbs energy of the \textbf{oxide}? (e.g., Cu$_2$O at $X_O = 1/3$)
    \item What is the Gibbs energy of \textbf{O$_2$ gas}?
\end{enumerate}

It then subtracts reactants from products to get $\Delta G$ of formation, writes one number per oxide per temperature to the CSV, and moves on. There is no curve fitting, no interpolation, and no thermodynamic modeling on our part. The science is entirely inside Thermo-Calc's TCOX14 database, which contains assessed model parameters fitted to decades of experimental calorimetry, EMF measurements, and phase equilibrium data. Our script is a glorified database query.

The equivalent GUI workflow would be: Property Diagram $\rightarrow$ select binary system (e.g., Cu-O) $\rightarrow$ set $X(O) = 0.333$ $\rightarrow$ plot $GM$ vs $T$ $\rightarrow$ read off values. Repeat for each oxide. Export to Excel. Make a chart. That takes roughly 45 minutes of clicking for all 7 oxides. The TC-Python script automates the same process in $\sim$5 minutes of runtime.

\subsection{What the Notebook Does with the Data}

The Marimo notebook reads the CSV and does three things:

\begin{enumerate}
    \item Reads the \texttt{dG\_[oxide]\_per\_O2} column for each oxide
    \item Divides by 1000 to convert J $\rightarrow$ kJ
    \item Calls \texttt{plt.plot(T\_C, dG\_vals)} to plot temperature vs.\ $\Delta G$
\end{enumerate}

That is the entire analysis. Matplotlib's \texttt{plot()} connects each consecutive pair of data points with a straight line segment. At 50~K spacing, the 31 dots are close enough that the result looks like a smooth, continuous curve.

The interactive controls (temperature slider, oxide selector) are cosmetic. The slider marks a vertical line at the selected temperature and the selector filters which oxides to display. Neither changes the underlying data.

\textbf{Note on the slider:} The Marimo slider uses 25\textdegree{}C steps from 500 to 1700\textdegree{}C, giving 49 positions. This controls only the vertical temperature marker, not the data resolution. The actual data is 31 points at 50~K spacing from 500 to 2000~K.

\subsection{Resolution: Why 31 Points Is Sufficient}

The 50~K step size was chosen for convenience, not necessity. To increase resolution, change one line in \texttt{extract\_oxide\_gibbs.py}:

\begin{lstlisting}
# Line 24 - current setting (31 points, ~5 min runtime)
T_STEP = 50

# Higher resolution (301 points, ~10 min runtime)
T_STEP = 5

# Maximum resolution (1501 points, excessive)
T_STEP = 1
\end{lstlisting}

The curves would be indistinguishable at any of these resolutions. This is because $\Delta G$ vs.\ $T$ for oxide formation is nearly linear, a consequence of the Gibbs-Helmholtz relation:

\begin{equation}
\Delta G^\circ = \Delta H^\circ - T\Delta S^\circ
\end{equation}

$\Delta H^\circ$ and $\Delta S^\circ$ are approximately constant over a given temperature range (they change slowly due to heat capacity). So $\Delta G$ vs.\ $T$ is nearly a straight line. The slight curvature comes from $C_p$ effects and phase transitions (for example, Cu$_2$O melting at 1235\textdegree{}C produces a visible slope change). 31 points at 50~K spacing captures this shape completely. The interpolation error between our connected line segments and a ``true'' continuous curve is less than 1~kJ/mol~O$_2$, which is meaningless when the stability gaps we are examining are $\sim$800~kJ.

\subsection{Is This a ``Real'' Ellingham Diagram?}

Yes. The data points are exactly the same numbers Thermo-Calc would produce in the GUI. We query the same TCOX14 database, the same equilibrium solver, and the same thermodynamic models. The only difference is resolution: we computed 31 points, whereas the GUI's Property Diagram calculator would compute several hundred. But as shown above, this does not affect the result.

Every textbook Ellingham diagram is constructed the same way. Ellingham's original 1944 paper was plotted from tabulated data points connected by line segments. There is no closed-form analytical function that produces a continuous Ellingham curve. Our version uses a commercial thermodynamic database instead of JANAF tables, but the method is identical.

\subsection{CSV Output Format}

The output CSV contains 31 rows (one per temperature) and 44 columns. For each oxide, the following columns are recorded (shown here using Cu$_2$O as an example):

\begin{table}[h]
\centering
\caption{CSV column descriptions (repeated for each of 7 oxides)}
\renewcommand{\arraystretch}{1.3}
\setlength{\arrayrulewidth}{0.8pt}
\begin{tabular}{|l|l|r|}
\hline
\rowcolor[HTML]{595959}
\textcolor{white}{\textbf{Column}} & \textcolor{white}{\textbf{Description}} & \textcolor{white}{\textbf{Example (1000 K)}} \\
\hline
\texttt{T\_K} & Temperature in Kelvin & 1000 \\
\hline
\rowcolor[HTML]{E0E0E0}
\texttt{T\_C} & Temperature in Celsius & 726.85 \\
\hline
\texttt{GM\_Cu2O} & Gibbs energy of Cu$_2$O phase, per mole atoms (J) & $-99{,}489$ \\
\hline
\rowcolor[HTML]{E0E0E0}
\texttt{GM\_system\_Cu2O} & Total system energy at that composition (J) & $-99{,}436$ \\
\hline
\texttt{G\_metal\_Cu2O} & Gibbs energy of pure Cu reference (J) & $-46{,}339$ \\
\hline
\rowcolor[HTML]{E0E0E0}
\texttt{dG\_Cu2O\_per\_O2} & \textbf{Final result:} $\Delta G_f^\circ$ in J/mol O$_2$ & $-190{,}813$ \\
\hline
\texttt{phases\_Cu2O} & Stable phases found by TC & CUPRITE\#1;FCC\_A1\#1 \\
\hline
\rowcolor[HTML]{E0E0E0}
\texttt{oxide\_phase\_Cu2O} & Phase used for the calculation & CUPRITE\#1 \\
\hline
\end{tabular}
\end{table}

This pattern repeats for all 7 oxides (Cu$_2$O, CuO, Al$_2$O$_3$, MgO, SiO$_2$, TiO$_2$, FeO). The only column the Marimo notebook reads is \texttt{dG\_[oxide]\_per\_O2}. Everything else is retained for traceability so the calculation can be verified at any step.

% ============================================================================
\section{Results}
% ============================================================================

\subsection{Gibbs Energies at 1000 K}

\begin{table}[h]
\centering
\caption{Gibbs energy of formation per mole O$_2$ at 1000 K (727°C)}
\begin{tabular}{lrll}
\toprule
\textbf{Oxide} & \textbf{$\Delta G_f^\circ$ (kJ/mol O$_2$)} & \textbf{Phase} & \textbf{Stability} \\
\midrule
MgO & $-986.8$ & HALITE\#1 & Most stable \\
Al$_2$O$_3$ & $-907.5$ & CORUNDUM\#1 & \\
TiO$_2$ & $-760.2$ & RUTILE\#1 & \\
SiO$_2$ & $-730.2$ & QUARTZ\#1 & \\
FeO & $-411.2$ & HALITE\#1 & \\
Cu$_2$O & $-190.8$ & CUPRITE\#1 & \\
CuO & $-132.0$ & CUO\#1 & Least stable \\
\bottomrule
\end{tabular}
\end{table}

\subsection{Key Finding}

\begin{keybox}[Thermodynamic Conclusion]
\textbf{Cu$_2$O is the LEAST stable oxide} among all candidates, with $\Delta G_f^\circ$ approximately 800 kJ/mol O$_2$ less negative than MgO or Al$_2$O$_3$.

This means copper \textbf{cannot} reduce these ceramics:
\[
\text{Cu} + \text{Al}_2\text{O}_3 \nrightarrow \text{Cu}_2\text{O} + \text{Al} \quad (\Delta G_{rxn} \gg 0)
\]

The thermodynamic driving force is approximately +800 kJ/mol O$_2$ in the \textit{wrong} direction.
\end{keybox}

\subsection{Physical Stability: Melting Points}

Cu oxides also melt at lower temperatures than the ceramic candidates:

\begin{table}[h]
\centering
\caption{Melting points and phase behavior at steelmaking temperatures}
\begin{tabular}{lrl}
\toprule
\textbf{Material} & \textbf{Melting Point (°C)} & \textbf{Status at 1500°C} \\
\midrule
Cu$_2$O & 1235 & \textbf{Liquid} \\
CuO & 1326 & \textbf{Liquid} \\
FeO & 1377 & \textbf{Liquid} \\
SiO$_2$ & 1713 & Solid \\
TiO$_2$ & 1843 & Solid \\
Al$_2$O$_3$ & 2072 & Solid \\
MgO & 2852 & Solid \\
\bottomrule
\end{tabular}
\end{table}

% ============================================================================
\section{Implications for the Project}
% ============================================================================

\subsection{What This Tells Us}
\begin{enumerate}
    \item \textbf{Direct oxide reduction is thermodynamically impossible} --- Cu cannot ``steal'' oxygen from Al$_2$O$_3$, MgO, SiO$_2$, or TiO$_2$
    \item \textbf{Previous Cu capture observations} must involve different mechanisms
\end{enumerate}

\subsection{Alternative Cu Removal Mechanisms}

\begin{table}[h]
\centering
\caption{Possible mechanisms for Cu capture by ceramics}
\begin{tabular}{p{3.5cm}p{5.5cm}p{4cm}}
\toprule
\textbf{Mechanism} & \textbf{Description} & \textbf{Key Factor} \\
\midrule
Solid Solution & Cu dissolves into oxide lattice (substitutional or interstitial) & High temperature (entropy-driven) \\
\addlinespace
Spinel Formation & CuO + Al$_2$O$_3$ $\rightarrow$ CuAl$_2$O$_4$ & Requires Cu oxidation first \\
\addlinespace
Surface Adsorption & Cu atoms adsorb on ceramic particle surfaces & Surface area, porosity \\
\addlinespace
Physical Infiltration & Fe-Cu melt infiltrates porous ceramic (not Cu-selective) & Porosity, capillary pressure \\
\bottomrule
\end{tabular}
\end{table}

\subsection{Recommended Next Steps}
\begin{enumerate}
    \item Model Cu solubility in Al$_2$O$_3$ and MgO using Thermo-Calc
    \item Calculate CuAl$_2$O$_4$ spinel stability conditions
    \item Compare predictions with last year's senior design experimental results
    \item Design experiments to test the dominant mechanism
\end{enumerate}

% ============================================================================
\section{Step-by-Step: Running on OSU Lab Machines}
% ============================================================================

This section documents exactly how the extraction was performed on the OSU ETS Virtual Machine.

\subsection{Accessing the Virtual Machine}

\begin{enumerate}
    \item Go to \url{https://ets.engineering.osu.edu/}
    \item Log in with OSU credentials
    \item Select a Windows VM with Thermo-Calc installed
    \item Open \textbf{File Explorer} and navigate to your U: drive
\end{enumerate}

\subsection{Getting the Code from GitHub}

Since Git is not installed on the lab VMs, download the repository as a ZIP file:

\begin{enumerate}
    \item Open a browser and go to: \url{https://github.com/dimascad/honda-calphad}
    \item Click the green \textbf{``Code''} button
    \item Select \textbf{``Download ZIP''}
    \item Extract the ZIP to your U: drive, e.g., \texttt{U:\textbackslash 4381\textbackslash honda-calphad\textbackslash}
\end{enumerate}

\subsection{Opening the Command Prompt}

\begin{enumerate}
    \item Press \texttt{Win + R}, type \texttt{cmd}, press Enter
    \item Navigate to the script directory:
\end{enumerate}

\begin{lstlisting}
cd U:\4381\honda-calphad\simulations\tcpython
\end{lstlisting}

\subsection{Running the TC-Python Script}

Thermo-Calc includes its own Python distribution. Use this exact path:

\begin{lstlisting}
"C:\Program Files\Thermo-Calc\2025b\python\python.exe" extract_oxide_gibbs.py
\end{lstlisting}

\textbf{Important:} The quotes around the path are required because of the spaces in ``Program Files''.

\subsection{Expected Output}

The script takes approximately 4--5 minutes to complete. You will see:

\begin{lstlisting}[basicstyle=\ttfamily\footnotesize]
======================================================================
TC-Python: Oxide Formation Energies for Ellingham Diagram
======================================================================
Temperature range: 500-2000 K (31 points)

Connected to Thermo-Calc

--- Getting O2 reference energies ---
  O2 reference at 1000K: -220766.0 J/mol-O2

--- Processing Cu2O ---
  Getting CU reference...
  Getting Cu2O oxide phase...
  Completed: 31/31 temperatures
  dG at 1000K: -190.8 kJ/mol O2
  Stable phases: CUPRITE#1;FCC_A1#1
  Used for calc: CUPRITE#1

[... continues for each oxide ...]

======================================================================
Writing to ...\data\tcpython\raw\oxide_gibbs_energies.csv
Done! 31 rows written.
======================================================================
\end{lstlisting}

\subsection{Retrieving the Output}

The CSV file is saved to:
\begin{lstlisting}
U:\4381\honda-calphad\data\tcpython\raw\oxide_gibbs_energies.csv
\end{lstlisting}

To get the file back to your local machine:
\begin{itemize}
    \item \textbf{Option A:} Copy to OneDrive/Google Drive from the VM
    \item \textbf{Option B:} Email the CSV to yourself
    \item \textbf{Option C:} Download directly from the VM browser
\end{itemize}

\subsection{Troubleshooting}

\begin{table}[h]
\centering
\caption{Common issues and solutions}
\begin{tabular}{p{5cm}p{8cm}}
\toprule
\textbf{Issue} & \textbf{Solution} \\
\midrule
``python is not recognized'' & Use the full path with quotes: \texttt{"C:\textbackslash Program Files\textbackslash Thermo-Calc\textbackslash 2025b\textbackslash python\textbackslash python.exe"} \\
\addlinespace
``tc\_python not found'' & You must use Thermo-Calc's bundled Python, not system Python \\
\addlinespace
``TCOX14 not found'' & Check available databases with \texttt{check\_databases.py} first \\
\addlinespace
IONIC\_LIQ warnings & These are normal and don't affect results \\
\bottomrule
\end{tabular}
\end{table}

% ============================================================================
\section{Files and Reproduction}
% ============================================================================

\subsection{Files Generated}

\begin{table}[h]
\centering
\caption{Output files from this analysis}
\begin{tabular}{ll}
\toprule
\textbf{File} & \textbf{Description} \\
\midrule
\texttt{data/tcpython/raw/oxide\_gibbs\_energies.csv} & Raw data (31 temps $\times$ 7 oxides) \\
\texttt{data/tcpython/ellingham\_diagram\_tcox14.png} & Static plot (PNG) \\
\texttt{data/tcpython/ellingham\_diagram\_tcox14.pdf} & Vector plot (PDF) \\
\texttt{simulations/notebooks/ellingham\_diagram.py} & Interactive Marimo notebook \\
\texttt{simulations/tcpython/extract\_oxide\_gibbs.py} & TC-Python extraction script \\
\bottomrule
\end{tabular}
\end{table}

\subsection{How to Reproduce}

\textbf{On OSU ETS Virtual Machine:}
\begin{lstlisting}
cd U:\4381\honda-calphad\simulations\tcpython
"C:\Program Files\Thermo-Calc\2025b\python\python.exe" extract_oxide_gibbs.py
\end{lstlisting}

Output: \texttt{data/tcpython/raw/oxide\_gibbs\_energies.csv}

\subsection{Interactive Notebook}

The Marimo notebook can be accessed via GitHub import:
\begin{center}
\url{https://github.com/dimascad/honda-calphad/blob/main/simulations/notebooks/ellingham_diagram.py}
\end{center}

% ============================================================================
\section{References}
% ============================================================================

\begin{enumerate}
    \item Thermo-Calc Software, \textit{TCOX14 Database Documentation}, 2024.
    \item Ellingham, H.J.T. (1944). ``Reducibility of oxides and sulphides in metallurgical processes.'' \textit{J. Soc. Chem. Ind.} 63: 125--133.
    \item Gaskell, D.R. \textit{Introduction to the Thermodynamics of Materials}, 5th ed. Taylor \& Francis, 2008.
    \item NIST-JANAF Thermochemical Tables, \url{https://janaf.nist.gov/}
\end{enumerate}

\vfill
\begin{center}
\textcolor{gray}{\textit{Document generated: February 4, 2026}}
\end{center}

\end{document}
